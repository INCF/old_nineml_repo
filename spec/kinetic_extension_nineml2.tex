\documentclass[draftspec]{ninemlspec}
\usepackage{microtype}
\usepackage{pbox}
\usepackage{multirow}
\usepackage{multicol}
\usepackage{float}
%% ============================================================================
%% Description:  Documentation for \lq\lq{}The NineML Specification Document\rq\rq{}
%% Authors: Thomas G. Close <tclose@oist.jp>, Ivan Raikov <raikov@oist.jp>, Andrew P. Davison <davison@unic.cnrs-gif.fr>
%% Organization: Okinawa Institute of Science and Technology Graduate University, Centre National de la Recherche Scientifique
%% Date created: October 2014  <---- should probably be some date in 2010, 2011 or so...
%% https://github.com/INCF/nineml/master/spec/specification.tex
%%
%% Copyright (C) 2014 Okinawa Institute of Science and Technology Graduate University, Centre National de la Recherche Scientifique
%%
%% ============================================================================

\newcommand{\incomplete}{\begin{center}\noindent{\Large\textcolor{incompletered}{\textbf{!! INCOMPLETE !!}}}\end{center}}

% Define misc. references
\newcommand{\identifier}{\typeDefRef{identifier\xspace}{sec:identifier}}
\newcommand{\URL}{\href{http://en.wikipedia.org/wiki/Uniform_resource_locator}{URL}\xspace}
\newcommand{\MathML}{\href{http://mathml.org}{MathML (http://mathml.org)}\xspace}

% Define Abstraction Layer element references

\newcommand{\Unit}{\defRef{\textbf{\class{Unit}}\xspace}{sec:Unit}}
\newcommand{\Dimension}{\defRef{\textbf{\class{Dimension}}\xspace}{sec:Dimension}}
\newcommand{\ComponentClass}{\defRef{\textbf{\class{ComponentClass}}\xspace}{sec:ComponentClass}}
\newcommand{\Dynamics}{\defRef{\defRef{\textbf{\class{Dynamics}}\xspace}{sec:Dynamics}}{sec:Dynamics}}
\newcommand{\Function}{\defRef{\textbf{\class{Function}}\xspace}{sec:Function}}
\newcommand{\BuiltInDistribution}{\defRef{\textbf{\class{BuiltInDistribution}}\xspace}{sec:BuiltInDistribution}}
\newcommand{\ConnectionRule}{\defRef{\textbf{\class{ConnectionRule}}\xspace}{sec:ConnectionRule}}
\newcommand{\ConnectCondition}{\defRef{\textbf{\class{ConnectCondition}}\xspace}{sec:ConnectCondition}}
\newcommand{\ExplicitConnections}{\defRef{\textbf{\class{ExplicitConnections}}\xspace}{sec:ExplicitConnections}}
\newcommand{\SourceIndices}{\defRef{\textbf{\class{SourceIndices}}\xspace}{sec:SourceIndices}}
\newcommand{\DestinationIndices}{\defRef{\textbf{\class{DestinationIndices}}\xspace}{sec:DestinationIndices}}
\newcommand{\SelectConnections}{\defRef{\textbf{\class{SelectConnections}}\xspace}{sec:SelectConnections}}
\newcommand{\Number}{\defRef{\textbf{\class{Number}}\xspace}{sec:Number}}
\newcommand{\Preference}{\defRef{\textbf{\class{Preference}}\xspace}{sec:Preference}}
\newcommand{\MathInline}{\defRef{\textbf{\class{MathInline}}\xspace}{sec:MathInline}}
\newcommand{\Piecewise}{\defRef{\textbf{\class{Piecewise}}\xspace}{sec:Piecewise}}
\newcommand{\Piece}{\defRef{\textbf{\class{Piece}}\xspace}{sec:Piece}}
\newcommand{\Expression}{\defRef{\textbf{\class{Expression}}\xspace}{sec:Expression}}
\newcommand{\Condition}{\defRef{\textbf{\class{Condition}}\xspace}{sec:Condition}}
\newcommand{\Otherwise}{\defRef{\textbf{\class{Otherwise}}\xspace}{sec:Otherwise}}
\newcommand{\KineticState}{\defRef{\textbf{\class{KineticState}}\xspace}{sec:KineticState}}
\newcommand{\StateAssignment}{\defRef{\textbf{\class{StateAssignment}}\xspace}{sec:StateAssignment}}
\newcommand{\TimeDerivative}{\defRef{\textbf{\class{TimeDerivative}}\xspace}{sec:TimeDerivative}}
\newcommand{\Alias}{\defRef{\textbf{\class{Alias}}\xspace}{sec:Alias}}
\newcommand{\Constant}{\defRef{\textbf{\class{Constant}}\xspace}{sec:Constant}}
\newcommand{\RandomVariable}{\defRef{\textbf{\class{RandomVariable}}\xspace}{sec:RandomVariable}}
\newcommand{\Argument}{\defRef{\textbf{\class{Argument}}\xspace}{sec:Argument}}
\newcommand{\StandardLibrary}{\defRef{\textbf{\class{StandardLibrary}}\xspace}{sec:StandardLibrary}}
\newcommand{\Regime}{\defRef{\textbf{\class{Regime}}\xspace}{sec:Regime}}
\newcommand{\Trigger}{\defRef{\textbf{\class{Trigger}}\xspace}{sec:Trigger}}
\newcommand{\EventOut}{\defRef{\textbf{\class{EventOut}}\xspace}{sec:EventOut}}
\newcommand{\OnEvent}{\defRef{\textbf{\class{OnEvent}}\xspace}{sec:OnEvent}}
\newcommand{\OnCondition}{\defRef{\textbf{\class{OnCondition}}\xspace}{sec:OnCondition}}
\newcommand{\Parameter}{\defRef{\textbf{\class{Parameter}}\xspace}{sec:Parameter}}
\newcommand{\AnalogSendPort}{\defRef{\textbf{\class{AnalogSendPort}}\xspace}{sec:AnalogSendPort}}


\newcommand{\EventSendPort}{\defRef{\textbf{\class{EventSendPort}}\xspace}{sec:EventSendPort}}
\newcommand{\AnalogReceivePort}{\defRef{\textbf{\class{AnalogReceivePort}}\xspace}{sec:AnalogReceivePort}}
\newcommand{\AnalogReducePort}{\defRef{\textbf{\class{AnalogReducePort}}\xspace}{sec:AnalogReducePort}}
\newcommand{\AnalogArrayPort}{\defRef{\textbf{\class{AnalogArrayPort}}\xspace}{sec:AnalogArrayPort}}
\newcommand{\EventReceivePort}{\defRef{\textbf{\class{EventReceivePort}}\xspace}{sec:EventReceivePort}}
\newcommand{\PropertySendPort}{\defRef{\textbf{\class{PropertySendPort}}\xspace}{sec:PropertySendPort}}
\newcommand{\PropertyReceivePort}{\defRef{\textbf{\class{PropertyReceivePort}}\xspace}{sec:PropertyReceivePort}}
\newcommand{\IndexSendPort}{\defRef{\textbf{\class{IndexSendPort}}\xspace}{sec:IndexSendPort}}
\newcommand{\IndexReceivePort}{\defRef{\textbf{\class{IndexReceivePort}}\xspace}{sec:IndexReceivePort}}
\newcommand{\Annotations}{\defRef{\textbf{\class{Annotations}}\xspace}{sec:Annotations}}

% Define User Layer element references

\newcommand{\Component}{\defRef{\textbf{\class{Component}}\xspace}{sec:Component}}
\newcommand{\Property}{\defRef{\textbf{\class{Property}}\xspace}{sec:Property}}
\newcommand{\DerivedProperty}{\defRef{\textbf{\class{DerivedProperty}}\xspace}{sec:DerivedProperty}}
\newcommand{\Quantity}{\defRef{\textbf{\class{Quantity}}\xspace}{sec:Quantity}}
\newcommand{\SingleValue}{\defRef{\textbf{\class{SingleValue}}\xspace}{sec:SingleValue}}
\newcommand{\ExternalArrayValue}{\defRef{\textbf{\class{ExternalArrayValue}}\xspace}{sec:ExternalArrayValue}}
\newcommand{\FromFunction}{\defRef{\textbf{\class{FromFunction}}\xspace}{sec:FromFunction}}
\newcommand{\ArrayValue}{\defRef{\textbf{\class{ArrayValue}}\xspace}{sec:ArrayValue}}
\newcommand{\ArrayValueRow}{\defRef{\textbf{\class{ArrayValueRow}}\xspace}{sec:ArrayValueRow}}
\newcommand{\ListColumn}{\defRef{\textbf{\class{ListColumn}}\xspace}{sec:ListColumn}}
\newcommand{\Definition}{\defRef{\textbf{\class{Definition}}\xspace}{sec:Definition}}
\newcommand{\Prototype}{\defRef{\textbf{\class{Prototype}}\xspace}{sec:Prototype}}
\newcommand{\Reference}{\defRef{\textbf{\class{Reference}}\xspace}{sec:Reference}}
\newcommand{\Constraint}{\defRef{\textbf{\class{Constraint}}\xspace}{sec:Constraint}}
\newcommand{\Cell}{\defRef{\textbf{\class{Cell}}\xspace}{sec:Cell}}
\newcommand{\Size}{\defRef{\textbf{\class{Size}}\xspace}{sec:Size}}
\newcommand{\AdditionalProperty}{\defRef{\textbf{\class{AdditionalProperty}}\xspace}{sec:AdditionalProperty}}
\newcommand{\Reaction}{\defRef{\textbf{\class{Reaction}}\xspace}{sec:Reaction}}
\newcommand{\Source}{\defRef{\textbf{\class{Source}}\xspace}{sec:Source}}
\newcommand{\Destination}{\defRef{\textbf{\class{Destination}}\xspace}{sec:Destination}}
\newcommand{\Connectivity}{\defRef{\textbf{\class{Connectivity}}\xspace}{sec:Connectivity}}
\newcommand{\Synapse}{\defRef{\textbf{\class{Synapse}}\xspace}{sec:Synapse}}
\newcommand{\Delay}{\defRef{\textbf{\class{Delay}}\xspace}{sec:Delay}}
\newcommand{\FromSource}{\defRef{\textbf{\class{FromSource}}\xspace}{sec:FromSource}}
\newcommand{\FromDestination}{\defRef{\textbf{\class{FromDestination}}\xspace}{sec:FromDestination}}
\newcommand{\FromSynapse}{\defRef{\textbf{\class{FromSynapse}}\xspace}{sec:FromSynapse}}
\newcommand{\FromIndex}{\defRef{\textbf{\class{FromIndex}}\xspace}{sec:FromIndex}}
\newcommand{\FromCellProperty}{\defRef{\textbf{\class{FromCellProperty}}\xspace}{sec:FromCellProperty}}
\newcommand{\FromAdditionalProperty}{\defRef{\textbf{\class{FromAdditionalProperty}}\xspace}{sec:FromAdditionalProperty}}
\newcommand{\Network}{\defRef{\textbf{\class{Network}}\xspace}{sec:Network}}
\newcommand{\Member}{\defRef{\textbf{\class{Member}}\xspace}{sec:Member}}
\newcommand{\Selection}{\defRef{\textbf{\class{Selection}}\xspace}{sec:Selection}}
\newcommand{\Concatenate}{\defRef{\textbf{\class{Concatenate}}\xspace}{sec:Concatenate}}
\newcommand{\Item}{\defRef{\textbf{\class{Item}}\xspace}{sec:Item}}

% Multi-component element references
\newcommand{\MultiComponentClass}{\defRef{\textbf{\class{MultiComponentClass}}\xspace}{sec:MultiComponentClass}}
\newcommand{\SubComponent}{\defRef{\textbf{\class{SubComponent}}\xspace}{sec:SubComponent}}
\newcommand{\PortExposure}{\defRef{\textbf{\class{PortExposure}}\xspace}{sec:PortExposure}}
\newcommand{\MetaParameter}{\defRef{\textbf{\class{MetaParameter}}\xspace}{sec:MetaParameter}}
\newcommand{\ReceiveConnection}{\defRef{\textbf{\class{ReceiveConnection}}\xspace}{sec:ReceiveConnection}}
\newcommand{\FromSister}{\defRef{\textbf{\class{FromSister}}\xspace}{sec:FromSister}}
\newcommand{\FromDomainID}{\defRef{\textbf{\class{FromDomainID}}\xspace}{sec:FromDomainID}}
\newcommand{\MultiComponent}{\defRef{\textbf{\class{MultiComponent}}\xspace}{sec:MultiComponent}}

% Multi-compartment element references

\newcommand{\MultiCompartmentClass}{\defRef{\textbf{\class{MultiCompartmentClass}}\xspace}{sec:MultiCompartmentClass}}
\newcommand{\MultiPortExposure}{\defRef{\textbf{\class{MultiPortExposure}}\xspace}{sec:MultiPortExposure}}
\newcommand{\Branches}{\defRef{\textbf{\class{Branches}}\xspace}{sec:Branches}}
\newcommand{\Mapping}{\defRef{\textbf{\class{Mapping}}\xspace}{sec:Mapping}}
\newcommand{\DomainClass}{\defRef{\textbf{\class{DomainClass}}\xspace}{sec:DomainClass}}
\newcommand{\Domain}{\defRef{\textbf{\class{Domain}}\xspace}{sec:Domain}}
\newcommand{\FromProximal}{\defRef{\textbf{\class{FromProximal}}\xspace}{sec:FromProximal}}
\newcommand{\FromDistal}{\defRef{\textbf{\class{FromDistal}}\xspace}{sec:FromDistal}}
\newcommand{\MultiCompartmental}{\defRef{\textbf{\class{MultiCompartmental}}\xspace}{sec:MultiCompartmental}}
\newcommand{\MultiCompartmentReaction}{\defRef{\textbf{\class{MultiCompartmentReaction}}\xspace}{sec:MultiCompartmentReaction}}
\newcommand{\CompartmentConnectivity}{\defRef{\textbf{\class{CompartmentConnectivity}}\xspace}{sec:CompartmentConnectivity}}

% Macros just for this document:

\newcommand{\ninemlpkg}{\texorpdfstring{%
    \textls[-25]{\textsc{NineMLSpec}}}{%
    \textsc{NineMLSpec}}\xspace}
\newcommand{\ninemlpkghead}{\texorpdfstring{%
    \textls[-50]{\textsc{NineMLSpec}}}{%
    \textsc{NineMLSpec}}\xspace}
\newcommand{\distURL}{https://github.com/INCF/nineml/tree/master/spec/specification.pdf}
\newcommand{\srcURL}{https://github.com/INCF/nineml/tree/master/spec/specification.tex}
\newcommand{\webURL}{https://github.com/INCF/nineml/tree/master/spec/specification.pdf}

\newcommand{\KineticEquation}{\defRef{\textbf{\class{KineticEquation}}\xspace}{sec:KineticEquation}}
\newcommand{\ForwardRate}{\defRef{\textbf{\class{ForwardRate}}\xspace}{sec:ForwardRate}}
\newcommand{\ReverseRate}{\defRef{\textbf{\class{ReverseRate}}\xspace}{sec:ReverseRate}}
\newcommand{\Conserve}{\defRef{\textbf{\class{Conserve}}\xspace}{sec:Conserve}}

% Custom latex listing style, for use with the listings package.  The default
% highlights far too many things, IMHO.  This keeps it simple and only adjusts
% the appearance of comments within listings.

\lstdefinelanguage{mylatex}{
  morekeywords={},%
  sensitive,%
  alsoother={0123456789$_},%$
  morecomment=[l]\%%
}[keywords,tex,comments]

\lstdefinestyle{latex}{language=mylatex}

% -----------------------------------------------------------------------------
% Start of document
% -----------------------------------------------------------------------------

\begin{document}

\packageTitle{Kinetic Extension to NineML (9ML) Specification}
\packageVersion{Version 2.0dev}
\packageVersionDate{ \today}

\pagestyle{empty}

\begin{center}
{\includegraphics[width=0.7\columnwidth]{figures/incf_new.png}}

\end{center}

\vspace*{0.5cm}

\noindent\rule{\columnwidth}{2pt}

\vspace*{0.75cm}

\begin{center}
\noindent{\Huge \bf NineML Kinetic Extension	}\\
\vspace{0.5cm}
\noindent{\LARGE \bf Specification}\\
\vspace{0.5cm}
\noindent{\large NineML Standardization Committee}\\
\vspace{0.5cm}
\noindent{\large Version: 2.0dev}
\end{center}

\vspace*{0.5cm}

\noindent\rule{\columnwidth}{2pt}

\vspace*{0.25cm}
\noindent{

{\Large\bf Editors: }
\begin{itemize}
\item Tom Close %probably the same whole list as NineML Spec?
\item Russell Jarvis
\end{itemize}

\vspace*{0.25cm}

\begin{normalsize}
\noindent \textbf{Acknowledgments:}\\\\
\noindent

\vspace*{0.5cm}

This document is under the Common Creative license BY-NC-SA:\\ http://creativecommons.org/licenses/by-nc-sa/3.0/

\vspace*{0.25cm}

{\flushright \includegraphics[width=3cm]{figures/by-nc-sa.png}}

\vspace*{0.5cm}

\noindent {\bf Date:} \today
\end{normalsize}
}

\title{Kinetic Extension to NineML Specification}

\newpage
\pagestyle{plain}

%\maketitlepage
%\maketableofcontents

% -----------------------------------------------------------------------------
\section{Introduction}

The purpose of NineML is to provide a simulator independent language for describing neuron relevant molecular, single cell, and neural network models. NineML is a declarative language which means that it is only necessary to describe the logic of the desired neural model, it is not necessary to provide implementation details. Because NineML provides a means to instantiate a model on any neural network simulator this will increase reproducibility of research.\\
\\
Because NEURON is a dominant simulator within the field of neuroscience NineML must be able to translate a complete model written in NMODL into an appropriate NineML format. NineML must also be able to translate appropriately formatted NineML model back to NMODL code. To successfully translate NMODL the KINETICs BLOCK of an NMODL mechanism must also be translated and it is this process that is described herein.

%insert paragraph 4 here.

%The Dynamics block represents the internal mechanisms governing the behaviour of the component. These
%dynamics are based on ordinary differential equations (ODE) but may contain non-linear transitions between
%different ODE regimes. The regime graph (e.g. Figure 2) must contain at least one Regime element, and contain
%no regime islands. At any given time, a component will be in a single regime, and can change which regime it is in
%through transitions.



% -----------------------------------------------------------------------------
%\vspace{-12.5pc} % A bit of a hack to reverse the vspace added by the Appendix name


\subsection{Scope}

The purpose of the kinetic extension to NINEML is to provide a convenient means of converting the KINETIC scheme block in a model described by NMODL to the NINEML language. The kinetic extension must also provided a means of converting from the NINEML language back into the KINETIC block scheme specified by NMODL. Ultimately the kinetic scheme should be able to handle discontinuous state transitions.



\subsection{KINETIC block}

The NMODL translator converts the content of the kinetic scheme block into a family of Ordinary Differential Equations. An equivalent but less readable method of doing this would be to declare the appropriate differential equations to be solved in a NEURON DERIVATIVE block\cite{carnevale2006neuron}

The concept of conservation of matter is central in calculating the fraction of ion channels that occupy a particular state. This is because ion transition from one state equals ion transition into another state.

The kinetics block can call a procedure such as { \it rates()}
The rates function solves the equilibrium constants, using time constants which may have been derived theoretically.



\section{NINEML Components}


The \Dynamics block represents the \emph{internal} mechanisms
governing the behaviour of the component. These dynamics are based on ordinary differential equations (ODE) but may contain non-linear transitions between
different ODE regimes. The regime graph (e.g. \ref{fig:simple_regime_graph}) must contain at least one \Regime element, and contain no regime islands. At any given time, a component will be in a single regime, and can change which regime it is in through transitions.

\note{\Alias objects are defined in Dynamics blocks, \emph{not} \Regime blocks. This means that aliases are the same across all regimes.}


\subsection{KineticState}
\label{sec:KineticState}

\begin{table}[H]
  \begin{edtable}{tabular}{llr}
    \toprule
    \multicolumn{3}{c}{\parbox{0.55\linewidth}{\center\textbf{KineticState Structure}}}\\
    \toprule
    \em{Attribute name} & \em{Type/Format} & \em{Required} \\
    \midrule
    name & \identifier & yes\\
    dimension & \Dimension{}@name & yes\\
    \bottomrule
  \end{edtable}
\end{table}

The internal state of a component is defined by a set of state variables
 -- variables that can change either continuously or discontinuously as a
function of time.

The value of a \KineticState can change in two ways:
\begin{quote}
\begin{itemize}
\item continuously through \TimeDerivative elements (in {\Regime} elements),
which define how the {\KineticState} evolves over time, e.g.
$dX/dt=1-X$.
\item discretely through \StateAssignment (in \OnCondition or \OnEvent transition elements),
which make discrete changes to a \KineticState value, e.g. $X = X + 1$.
\end{itemize}
\end{quote}

Only continuous changes are relevant to NineML KINETICs scheme.



\subsection{TimeDerivative}
\label{sec:TimeDerivative}

\begin{table}[H]
  \begin{edtable}{tabular}{llr}
    \toprule
    \multicolumn{3}{c}{\parbox{0.55\linewidth}{\center\textbf{TimeDerivative Structure}}}\\
    \toprule
    \em{Attribute name} & \em{Type/Format} & \em{Required} \\
    \midrule
    variable & \KineticState{}@name & yes\\
    \midrule
    \em{Element type} & \em{Multiplicity} & \em{Required} \\
    \midrule
    \MathInline & singleton & yes \\ 
    \bottomrule
  \end{edtable}
\end{table}

\TimeDerivative elements contain a mathematical expression for the right-hand side of the ODE
\begin{equation}
\frac{\mathrm{d} variable}{\mathrm{d} t} = expression
\end{equation}
which can contain of references to any combination of \KineticState, \Parameter, \AnalogReceivePort, \AnalogReducePort and \Alias. Therefore, only one \TimeDerivative element is allowed per \KineticState per \Regime. If a {\TimeDerivative} for a \KineticState is not defined in a \Regime, it is assumed to be zero.

\subsubsection{Variable attribute}
Each \TimeDerivative requires a \textit{variable} attribute. This should refer to the name of a \KineticState in the \ComponentClass. Only one \TimeDerivative is allowed per \textit{variable} in each \Regime.




\subsection{Kinetic}
%\label{sec:ComponentClass}

The motivation for Kinetic Component Class extension to nineml is to provide an opportunity for electro-physiologists to express kinetic schemes in a notation that is consistent with their field.
%Ie flattening of the tree is facilitated by mapping onto a dynamics component.
The Kinetic Component Class can be flattened into a \Dynamics component. A kinetic scheme usually contains a multiple number of states, and a multiple number of reactions. A reaction is a process that facilitates the transition of a molecular entity between states. %Within the Kinetics component the user of the Kinetics component can instantiate any number of KineticStates, and Reactions. 

\begin{table}[H]
  \begin{edtable}{tabular}{llr}
    \toprule
    \multicolumn{3}{c}{\parbox{0.55\linewidth}{\center\textbf{Kinetic Structure}}}\\
    \toprule
    \em{Element type} & \em{Multiplicity} & \em{Required} \\
    \midrule
    \KineticState & set & yes \\
    \Reaction & set & yes \\
    \Constraint & set & no \\
    \Alias & set & no \\
    \Constant & set & no \\
    \bottomrule
  \end{edtable}
\end{table}


\subsection{KineticState}
\label{sec:KineticState}

Each Kinetic scheme is able to accommodate an arbitrary number of states. 

\begin{table}[H]
  \begin{edtable}{tabular}{llr}
    \toprule
    \multicolumn{3}{c}{\parbox{0.55\linewidth}{\center\textbf{Kinetic Structure}}}\\
    \toprule
    \em{Attribute name} & \em{Type/Format} & \em{Required} \\
    \midrule
    name & \identifier & yes\\
    \bottomrule
  \end{edtable}
\end{table}

\subsubsection{Name attribute}


\subsection{Reaction}
\label{sec:Reaction}
Because the number of possible ion channel states is variable the number of reactions between states must also be variable, however it is expected that the number of possible reactions should be at least one less than the number of possible states. Reactions have a from and a to attribute, which describes which state variable, they are moving from and which state varible the moving to.

\begin{table}[H]
  \begin{edtable}{tabular}{llr}
    \toprule
    \multicolumn{3}{c}{\parbox{0.55\linewidth}{\center\textbf{Reaction Structure}}}\\
    \toprule
    \em{Attribute name} & \em{Type/Format} & \em{Required} \\
    \midrule
    from & \KineticState{}@name & yes\\
    to & \KineticState{}@name & yes\\
    \midrule
    \em{Element type} & \em{Multiplicity} & \em{Required} \\
    \midrule
    \ForwardRate & singleton & yes\\
    \ReverseRate & singleton & yes\\
    \bottomrule
  \end{edtable}
\end{table}

\subsubsection{From attribute}

Attribute name Type/Format Required
name identifier yes

\subsubsection{To attribute}

Attribute name Type/Format Required
name identifier yes


\subsection{ForwardRate}
\label{sec:ForwardRate}
The forward rate describes the rate of change of a molecular entity from one state to another  an example might be ion channels as they move from a closed state to an open state. At most one forward rate equation and one reverse rate per Reaction.  
\begin{table}[H]
  \begin{edtable}{tabular}{llr}
    \toprule
    \multicolumn{3}{c}{\parbox{0.55\linewidth}{\center\textbf{ForwardRate Structure}}}\\
    \toprule
    \em{Element type} & \em{Multiplicity} & \em{Required} \\
    \midrule
    \MathInline & singleton & yes\\
    \bottomrule
  \end{edtable}
\end{table}


\subsection{ReverseRate}
\label{sec:ReverseRate}
The reverse rate describes that rate of change for the reverse process described in ForwardRate.

\begin{table}[H]
  \begin{edtable}{tabular}{llr}
    \toprule
    \multicolumn{3}{c}{\parbox{0.55\linewidth}{\center\textbf{ReverseRate Structure}}}\\
    \toprule
    \em{Element type} & \em{Multiplicity} & \em{Required} \\
    \midrule
    \MathInline & singleton & yes\\
    \bottomrule
  \end{edtable}
\end{table}


\subsection{Constraint}
\label{sec:Constraint}

The right hand side of the kinetics equation must be stated in a format where left hand side is equal to unity. If a kinetic equation is expressed with a non unity right hand side they may require a basic manipulation. At most the user can specify only one constraint equation per Kinetic Component.

The concept of conservation of matter is central in calculating the fraction molecular entities that occupy a particular state. This is because molecular transition out of one state equals molecular transition into another state.

\begin{table}[H]
  \begin{edtable}{tabular}{llr}
    \toprule
    \multicolumn{3}{c}{\parbox{0.55\linewidth}{\center\textbf{Constraint Structure}}}\\
    \toprule
    \em{Element type} & \em{Multiplicity} & \em{Required} \\
    \midrule
    \MathInline & singleton & yes\\
    \bottomrule
  \end{edtable}
\end{table}


\subsection{Example use of kinetic scheme}.


\begin{lstlisting}[label=code:xmliaf2]
<?xml version='1.0' encoding='UTF-8'?>
<NineML xmlns="http://nineml.net/9ML/2.0">
  <ComponentClass name="Kinetic">
    <KineticState name C2> 
      <to> O </to>
      <from> C1 </from>
    </KineticState>
    <Alias name="K1">    
      <mathinline>  K1=exp(k2*(d2-v)-k1*(d1-v)) </MathInline>
    <Alias>
    <Alias name="K2">    
  	  <mathinline>  K2=exp(-k2*(d2-v)) </MathInline>
    <Alias>    
    <Reaction>
      <ForwardRate> <mathinline> kf2= K2/(tau2v)*(1+K2)) </mathinline> </ForwardRate>
      <ReverseRate> <mathinline> kb1=1/(tau1(v)*(1+K1)) </mathinline>  </ReverseRate>      
    </Reaction>
  
    <KineticState name = O> 
      <to> C2 </to>
      <from> C2  </from>
    </KineticState>
    <Alias name="K2">    
  	  <mathinline>  K2=exp(-k2*(d2-v)) </MathInline>
    <Alias>       
    <Reaction>
      <ForwardRate> <mathinline> kf2= K2/(tau2v)*(1+K2)) </mathinline> </ForwardRate>
      <ReverseRate> <mathinline> kb2=t/(tau2(v)+(1+K2) </mathinline> </ReverseRate>
    </Reaction>
    
    <KineticState name = C1> 
      <to> C2 </to>
      <from> C2  </from>
    </KineticState>  
      <Alias name="K1">
	  	<mathinline>  K1=exp(k2*(d2-v)-k1*(d1-v)) </MathInline> 
  	  </Alias>	
    <Reaction>
      <ForwardRate> <mathinline> kf1=k1/(tau1(v)*(1+K1)) </mathinline>  </ForwardRate>
      <ReverseRate> <mathinline> kb1=1/(tau1(v)*(1+K1)) </mathinline>  </ReverseRate>
    </Reaction> 
     
    <constraint> c1+c2+o=1 </constraint>
  <ComponentClass>  
      
	  	
	   
\end{lstlisting}[label=code:xmliaf2]


\clearpage
\bibliography{specification}

% -----------------------------------------------------------------------------
% End of document
% -----------------------------------------------------------------------------

\end{document}
