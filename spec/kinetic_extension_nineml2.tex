\documentclass[draftspec]{ninemlspec}
\usepackage{microtype}
\usepackage{pbox}
\usepackage{multirow}
\usepackage{multicol}
\usepackage{float}
%% ============================================================================
%% Description:  Documentation for \lq\lq{}The NineML Specification Document\rq\rq{}
%% Authors: Thomas G. Close <tclose@oist.jp>, Ivan Raikov <raikov@oist.jp>, Andrew P. Davison <davison@unic.cnrs-gif.fr>
%% Organization: Okinawa Institute of Science and Technology Graduate University, Centre National de la Recherche Scientifique
%% Date created: October 2014  <---- should probably be some date in 2010, 2011 or so...
%% https://github.com/INCF/nineml/master/spec/specification.tex
%%
%% Copyright (C) 2014 Okinawa Institute of Science and Technology Graduate University, Centre National de la Recherche Scientifique
%%
%% ============================================================================

\newcommand{\incomplete}{\begin{center}\noindent{\Large\textcolor{incompletered}{\textbf{!! INCOMPLETE !!}}}\end{center}}

% Define misc. references
\newcommand{\identifier}{\typeDefRef{identifier\xspace}{sec:identifier}}
\newcommand{\URL}{\href{http://en.wikipedia.org/wiki/Uniform_resource_locator}{URL}\xspace}
\newcommand{\MathML}{\href{http://mathml.org}{MathML (http://mathml.org)}\xspace}

% Define Abstraction Layer element references

\newcommand{\Unit}{\defRef{\textbf{\class{Unit}}\xspace}{sec:Unit}}
\newcommand{\Dimension}{\defRef{\textbf{\class{Dimension}}\xspace}{sec:Dimension}}
\newcommand{\ComponentClass}{\defRef{\textbf{\class{ComponentClass}}\xspace}{sec:ComponentClass}}
\newcommand{\Dynamics}{\defRef{\defRef{\textbf{\class{Dynamics}}\xspace}{sec:Dynamics}}{sec:Dynamics}}
\newcommand{\Function}{\defRef{\textbf{\class{Function}}\xspace}{sec:Function}}
\newcommand{\BuiltInDistribution}{\defRef{\textbf{\class{BuiltInDistribution}}\xspace}{sec:BuiltInDistribution}}
\newcommand{\ConnectionRule}{\defRef{\textbf{\class{ConnectionRule}}\xspace}{sec:ConnectionRule}}
\newcommand{\ConnectCondition}{\defRef{\textbf{\class{ConnectCondition}}\xspace}{sec:ConnectCondition}}
\newcommand{\ExplicitConnections}{\defRef{\textbf{\class{ExplicitConnections}}\xspace}{sec:ExplicitConnections}}
\newcommand{\SourceIndices}{\defRef{\textbf{\class{SourceIndices}}\xspace}{sec:SourceIndices}}
\newcommand{\DestinationIndices}{\defRef{\textbf{\class{DestinationIndices}}\xspace}{sec:DestinationIndices}}
\newcommand{\SelectConnections}{\defRef{\textbf{\class{SelectConnections}}\xspace}{sec:SelectConnections}}
\newcommand{\Number}{\defRef{\textbf{\class{Number}}\xspace}{sec:Number}}
\newcommand{\Preference}{\defRef{\textbf{\class{Preference}}\xspace}{sec:Preference}}
\newcommand{\MathInline}{\defRef{\textbf{\class{MathInline}}\xspace}{sec:MathInline}}
\newcommand{\Piecewise}{\defRef{\textbf{\class{Piecewise}}\xspace}{sec:Piecewise}}
\newcommand{\Piece}{\defRef{\textbf{\class{Piece}}\xspace}{sec:Piece}}
\newcommand{\Expression}{\defRef{\textbf{\class{Expression}}\xspace}{sec:Expression}}
\newcommand{\Condition}{\defRef{\textbf{\class{Condition}}\xspace}{sec:Condition}}
\newcommand{\Otherwise}{\defRef{\textbf{\class{Otherwise}}\xspace}{sec:Otherwise}}
\newcommand{\KineticState}{\defRef{\textbf{\class{KineticState}}\xspace}{sec:KineticState}}
\newcommand{\StateAssignment}{\defRef{\textbf{\class{StateAssignment}}\xspace}{sec:StateAssignment}}
\newcommand{\TimeDerivative}{\defRef{\textbf{\class{TimeDerivative}}\xspace}{sec:TimeDerivative}}
\newcommand{\Alias}{\defRef{\textbf{\class{Alias}}\xspace}{sec:Alias}}
\newcommand{\Constant}{\defRef{\textbf{\class{Constant}}\xspace}{sec:Constant}}
\newcommand{\RandomVariable}{\defRef{\textbf{\class{RandomVariable}}\xspace}{sec:RandomVariable}}
\newcommand{\Argument}{\defRef{\textbf{\class{Argument}}\xspace}{sec:Argument}}
\newcommand{\StandardLibrary}{\defRef{\textbf{\class{StandardLibrary}}\xspace}{sec:StandardLibrary}}
\newcommand{\Regime}{\defRef{\textbf{\class{Regime}}\xspace}{sec:Regime}}
\newcommand{\Trigger}{\defRef{\textbf{\class{Trigger}}\xspace}{sec:Trigger}}
\newcommand{\EventOut}{\defRef{\textbf{\class{EventOut}}\xspace}{sec:EventOut}}
\newcommand{\OnEvent}{\defRef{\textbf{\class{OnEvent}}\xspace}{sec:OnEvent}}
\newcommand{\OnCondition}{\defRef{\textbf{\class{OnCondition}}\xspace}{sec:OnCondition}}
\newcommand{\Parameter}{\defRef{\textbf{\class{Parameter}}\xspace}{sec:Parameter}}
\newcommand{\AnalogSendPort}{\defRef{\textbf{\class{AnalogSendPort}}\xspace}{sec:AnalogSendPort}}


\newcommand{\EventSendPort}{\defRef{\textbf{\class{EventSendPort}}\xspace}{sec:EventSendPort}}
\newcommand{\AnalogReceivePort}{\defRef{\textbf{\class{AnalogReceivePort}}\xspace}{sec:AnalogReceivePort}}
\newcommand{\AnalogReducePort}{\defRef{\textbf{\class{AnalogReducePort}}\xspace}{sec:AnalogReducePort}}
\newcommand{\AnalogArrayPort}{\defRef{\textbf{\class{AnalogArrayPort}}\xspace}{sec:AnalogArrayPort}}
\newcommand{\EventReceivePort}{\defRef{\textbf{\class{EventReceivePort}}\xspace}{sec:EventReceivePort}}
\newcommand{\PropertySendPort}{\defRef{\textbf{\class{PropertySendPort}}\xspace}{sec:PropertySendPort}}
\newcommand{\PropertyReceivePort}{\defRef{\textbf{\class{PropertyReceivePort}}\xspace}{sec:PropertyReceivePort}}
\newcommand{\IndexSendPort}{\defRef{\textbf{\class{IndexSendPort}}\xspace}{sec:IndexSendPort}}
\newcommand{\IndexReceivePort}{\defRef{\textbf{\class{IndexReceivePort}}\xspace}{sec:IndexReceivePort}}
\newcommand{\Annotations}{\defRef{\textbf{\class{Annotations}}\xspace}{sec:Annotations}}

% Define User Layer element references

\newcommand{\Component}{\defRef{\textbf{\class{Component}}\xspace}{sec:Component}}
\newcommand{\Property}{\defRef{\textbf{\class{Property}}\xspace}{sec:Property}}
\newcommand{\DerivedProperty}{\defRef{\textbf{\class{DerivedProperty}}\xspace}{sec:DerivedProperty}}
\newcommand{\Quantity}{\defRef{\textbf{\class{Quantity}}\xspace}{sec:Quantity}}
\newcommand{\SingleValue}{\defRef{\textbf{\class{SingleValue}}\xspace}{sec:SingleValue}}
\newcommand{\ExternalArrayValue}{\defRef{\textbf{\class{ExternalArrayValue}}\xspace}{sec:ExternalArrayValue}}
\newcommand{\FromFunction}{\defRef{\textbf{\class{FromFunction}}\xspace}{sec:FromFunction}}
\newcommand{\ArrayValue}{\defRef{\textbf{\class{ArrayValue}}\xspace}{sec:ArrayValue}}
\newcommand{\ArrayValueRow}{\defRef{\textbf{\class{ArrayValueRow}}\xspace}{sec:ArrayValueRow}}
\newcommand{\ListColumn}{\defRef{\textbf{\class{ListColumn}}\xspace}{sec:ListColumn}}
\newcommand{\Definition}{\defRef{\textbf{\class{Definition}}\xspace}{sec:Definition}}
\newcommand{\Prototype}{\defRef{\textbf{\class{Prototype}}\xspace}{sec:Prototype}}
\newcommand{\Reference}{\defRef{\textbf{\class{Reference}}\xspace}{sec:Reference}}
\newcommand{\Constraint}{\defRef{\textbf{\class{Constraint}}\xspace}{sec:Constraint}}
\newcommand{\Cell}{\defRef{\textbf{\class{Cell}}\xspace}{sec:Cell}}
\newcommand{\Size}{\defRef{\textbf{\class{Size}}\xspace}{sec:Size}}
\newcommand{\AdditionalProperty}{\defRef{\textbf{\class{AdditionalProperty}}\xspace}{sec:AdditionalProperty}}
\newcommand{\Reaction}{\defRef{\textbf{\class{Reaction}}\xspace}{sec:Reaction}}
\newcommand{\Source}{\defRef{\textbf{\class{Source}}\xspace}{sec:Source}}
\newcommand{\Destination}{\defRef{\textbf{\class{Destination}}\xspace}{sec:Destination}}
\newcommand{\Connectivity}{\defRef{\textbf{\class{Connectivity}}\xspace}{sec:Connectivity}}
\newcommand{\Synapse}{\defRef{\textbf{\class{Synapse}}\xspace}{sec:Synapse}}
\newcommand{\Delay}{\defRef{\textbf{\class{Delay}}\xspace}{sec:Delay}}
\newcommand{\FromSource}{\defRef{\textbf{\class{FromSource}}\xspace}{sec:FromSource}}
\newcommand{\FromDestination}{\defRef{\textbf{\class{FromDestination}}\xspace}{sec:FromDestination}}
\newcommand{\FromSynapse}{\defRef{\textbf{\class{FromSynapse}}\xspace}{sec:FromSynapse}}
\newcommand{\FromIndex}{\defRef{\textbf{\class{FromIndex}}\xspace}{sec:FromIndex}}
\newcommand{\FromCellProperty}{\defRef{\textbf{\class{FromCellProperty}}\xspace}{sec:FromCellProperty}}
\newcommand{\FromAdditionalProperty}{\defRef{\textbf{\class{FromAdditionalProperty}}\xspace}{sec:FromAdditionalProperty}}
\newcommand{\Network}{\defRef{\textbf{\class{Network}}\xspace}{sec:Network}}
\newcommand{\Member}{\defRef{\textbf{\class{Member}}\xspace}{sec:Member}}
\newcommand{\Selection}{\defRef{\textbf{\class{Selection}}\xspace}{sec:Selection}}
\newcommand{\Concatenate}{\defRef{\textbf{\class{Concatenate}}\xspace}{sec:Concatenate}}
\newcommand{\Item}{\defRef{\textbf{\class{Item}}\xspace}{sec:Item}}

% Multi-component element references
\newcommand{\MultiComponentClass}{\defRef{\textbf{\class{MultiComponentClass}}\xspace}{sec:MultiComponentClass}}
\newcommand{\SubComponent}{\defRef{\textbf{\class{SubComponent}}\xspace}{sec:SubComponent}}
\newcommand{\PortExposure}{\defRef{\textbf{\class{PortExposure}}\xspace}{sec:PortExposure}}
\newcommand{\MetaParameter}{\defRef{\textbf{\class{MetaParameter}}\xspace}{sec:MetaParameter}}
\newcommand{\ReceiveConnection}{\defRef{\textbf{\class{ReceiveConnection}}\xspace}{sec:ReceiveConnection}}
\newcommand{\FromSister}{\defRef{\textbf{\class{FromSister}}\xspace}{sec:FromSister}}
\newcommand{\FromDomainID}{\defRef{\textbf{\class{FromDomainID}}\xspace}{sec:FromDomainID}}
\newcommand{\MultiComponent}{\defRef{\textbf{\class{MultiComponent}}\xspace}{sec:MultiComponent}}

% Multi-compartment element references

\newcommand{\MultiCompartmentClass}{\defRef{\textbf{\class{MultiCompartmentClass}}\xspace}{sec:MultiCompartmentClass}}
\newcommand{\MultiPortExposure}{\defRef{\textbf{\class{MultiPortExposure}}\xspace}{sec:MultiPortExposure}}
\newcommand{\Branches}{\defRef{\textbf{\class{Branches}}\xspace}{sec:Branches}}
\newcommand{\Mapping}{\defRef{\textbf{\class{Mapping}}\xspace}{sec:Mapping}}
\newcommand{\DomainClass}{\defRef{\textbf{\class{DomainClass}}\xspace}{sec:DomainClass}}
\newcommand{\Domain}{\defRef{\textbf{\class{Domain}}\xspace}{sec:Domain}}
\newcommand{\FromProximal}{\defRef{\textbf{\class{FromProximal}}\xspace}{sec:FromProximal}}
\newcommand{\FromDistal}{\defRef{\textbf{\class{FromDistal}}\xspace}{sec:FromDistal}}
\newcommand{\MultiCompartmental}{\defRef{\textbf{\class{MultiCompartmental}}\xspace}{sec:MultiCompartmental}}
\newcommand{\MultiCompartmentReaction}{\defRef{\textbf{\class{MultiCompartmentReaction}}\xspace}{sec:MultiCompartmentReaction}}
\newcommand{\CompartmentConnectivity}{\defRef{\textbf{\class{CompartmentConnectivity}}\xspace}{sec:CompartmentConnectivity}}

% Macros just for this document:

\newcommand{\ninemlpkg}{\texorpdfstring{%
    \textls[-25]{\textsc{NineMLSpec}}}{%
    \textsc{NineMLSpec}}\xspace}
\newcommand{\ninemlpkghead}{\texorpdfstring{%
    \textls[-50]{\textsc{NineMLSpec}}}{%
    \textsc{NineMLSpec}}\xspace}
\newcommand{\distURL}{https://github.com/INCF/nineml/tree/master/spec/specification.pdf}
\newcommand{\srcURL}{https://github.com/INCF/nineml/tree/master/spec/specification.tex}
\newcommand{\webURL}{https://github.com/INCF/nineml/tree/master/spec/specification.pdf}

\newcommand{\KineticEquation}{\defRef{\textbf{\class{KineticEquation}}\xspace}{sec:KineticEquation}}
\newcommand{\ForwardRate}{\defRef{\textbf{\class{ForwardRate}}\xspace}{sec:ForwardRate}}
\newcommand{\ReverseRate}{\defRef{\textbf{\class{ReverseRate}}\xspace}{sec:ReverseRate}}
\newcommand{\Conserve}{\defRef{\textbf{\class{Conserve}}\xspace}{sec:Conserve}}

% Custom latex listing style, for use with the listings package.  The default
% highlights far too many things, IMHO.  This keeps it simple and only adjusts
% the appearance of comments within listings.

\lstdefinelanguage{mylatex}{
  morekeywords={},%
  sensitive,%
  alsoother={0123456789$_},%$
  morecomment=[l]\%%
}[keywords,tex,comments]

\lstdefinestyle{latex}{language=mylatex}

% -----------------------------------------------------------------------------
% Start of document
% -----------------------------------------------------------------------------

\begin{document}

\packageTitle{Kinetic Extension to NineML (9ML) Specification}
\packageVersion{Version 2.0dev}
\packageVersionDate{ \today}

\pagestyle{empty}

\begin{center}
{\includegraphics[width=0.7\columnwidth]{figures/incf_new.png}}

\end{center}

\vspace*{0.5cm}

\noindent\rule{\columnwidth}{2pt}

\vspace*{0.75cm}

\begin{center}
\noindent{\Huge \bf NineML Kinetic Extension	}\\
\vspace{0.5cm}
\noindent{\LARGE \bf Specification}\\
\vspace{0.5cm}
\noindent{\large NineML Standardization Committee}\\
\vspace{0.5cm}
\noindent{\large Version: 2.0dev}
\end{center}

\vspace*{0.5cm}

\noindent\rule{\columnwidth}{2pt}

\vspace*{0.25cm}
\noindent{

{\Large\bf Editors: }
\begin{itemize}
\item Alex Cope
\item Andrew P. Davison
\item Erik De Schutter
\item Ivan Raikov
\item Paul Richmond
\item Thomas G. Close
\item Russell Jarvis
\end{itemize}

\vspace*{0.25cm}

\begin{normalsize}
\noindent \textbf{Acknowledgments:}\\\\
\noindent

\vspace*{0.5cm}

This document is under the Common Creative license BY-NC-SA:\\ http://creativecommons.org/licenses/by-nc-sa/3.0/

\vspace*{0.25cm}

{\flushright \includegraphics[width=3cm]{figures/by-nc-sa.png}}

\vspace*{0.5cm}

\noindent {\bf Date:} \today
\end{normalsize}
}

\title{Kinetic Extension to NineML Specification}

\newpage
\pagestyle{plain}

%\maketitlepage
%\maketableofcontents

\subsection{Scope}

The main motivation for Kinetic Component Class extension to NineML is to provide an opportunity for electro-physiologists to express kinetic schemes in a notation that is consistent with their field. The Kinetic Component Class can be flattened into a \Dynamics component. Another purpose of the kinetic extension to NINEML is to provide a convenient means of converting the KINETIC scheme block in a model described by NMODL to the NINEML language. %The kinetic extension must also provided a means of converting from the NINEML language back into the KINETIC block scheme specified by NMODL. Ultimately the kinetic scheme should be able to handle discontinuous state transitions.

\subsection{KINETIC block}

The NMODL translator converts the content of the kinetic scheme block into a family of Ordinary Differential Equations. An equivalent but less readable method of doing this would be to declare the appropriate differential equations to be solved in a NEURON DERIVATIVE block \cite{carnevale2006neuron}.

\section{NINEML Components}

\subsection{Kinetic}
%\label{sec:ComponentClass}


%Ie flattening of the tree is facilitated by mapping onto a dynamics component.
The Kinetic component provides all of the information required to fully specify a Kinetic scheme. Because a kinetic scheme usually contains a multiple number of states, and a multiple number of reactions within the Kinetics component the user of the Kinetics component can instantiate any number of KineticStates, and Reactions. 

\begin{table}[H]
  \begin{edtable}{tabular}{llr}
    \toprule
    \multicolumn{3}{c}{\parbox{0.55\linewidth}{\center\textbf{Kinetic Structure}}}\\
    \toprule
    \em{Element type} & \em{Multiplicity} & \em{Required} \\
    \midrule
    \KineticState & set & yes \\
    \Reaction & set & yes \\
    \Constraint & set & no \\
    \Alias & set & no \\
    \Constant & set & no \\ %I think this should be singleton and required.
    \bottomrule
  \end{edtable}
\end{table}


\subsection{KineticState}
\label{sec:KineticState}
The KineticState container provides all of the information to fully describe one kinetic state variable that may constitute a greater kinetic scheme. %Each Kinetic scheme is able to accommodate an arbitrary number of states, and also a variable number of Reactions between states.

\begin{table}[H]
  \begin{edtable}{tabular}{llr}
    \toprule
    \multicolumn{3}{c}{\parbox{0.55\linewidth}{\center\textbf{KineticState}}}\\
    \toprule
    \em{Attribute name} & \em{Type/Format} & \em{Required} \\
    \midrule
    name & \identifier & yes\\
    \bottomrule
  \end{edtable}
\end{table}

\subsubsection{Name attribute}
Each KineticState requires a name attribute, which is a valid identifier and uniquely identifies the StateVariable from all other elements in the ComponentClass.

\subsubsection{Dimension attribute}
Each KineticState Variable requires a dimension attribute. This attribute specifies the dimension of the units of the quantities that KineticState is expected to be initialised and updated with and should refer to the name of a Dimension element in the global scope.

%ie in the example of ion channel states the units of a kinetic state variable would be in energy. 

\subsection{Reaction}
\label{sec:Reaction} A reaction is a process that facilitates the transition of a molecular entity between states. Reactions have a single to and single from attribute, which describes which state  they are transitioning from and which state they are transitioning into. The forward rate and reverse rate of chemical reactions are described inside the Reaction Component class.

\begin{table}[H]
  \begin{edtable}{tabular}{llr}
    \toprule
    \multicolumn{3}{c}{\parbox{0.55\linewidth}{\center\textbf{Reaction Structure}}}\\
    \toprule
    \em{Attribute name} & \em{Type/Format} & \em{Required} \\
    \midrule
    from & \KineticState{}@name & yes\\
    to & \KineticState{}@name & yes\\
    \midrule
    \em{Element type} & \em{Multiplicity} & \em{Required} \\
    \midrule
    \ForwardRate & singleton & yes\\
    \ReverseRate & singleton & yes\\
    \bottomrule
  \end{edtable}
\end{table}

\subsubsection{From attribute}

Attribute name Type/Format Required
name identifier yes

\subsubsection{To attribute}

Attribute name Type/Format Required 
name identifier yes


\subsection{ForwardRate}
\label{sec:ForwardRate}
The forward rate describes the rate of change of a molecular entity from one state to another.  An example of a forward reaction might be an ion channel protein as it moves from a closed state to an open state. %At most there can only be one forward rate equation and one reverse rate per Reaction.  
\begin{table}[H]
  \begin{edtable}{tabular}{llr}
    \toprule
    \multicolumn{3}{c}{\parbox{0.55\linewidth}{\center\textbf{ForwardRate Structure}}}\\
    \toprule
    \em{Element type} & \em{Multiplicity} & \em{Required} \\
    \midrule
    \MathInline & singleton & yes\\
    \bottomrule
  \end{edtable}
\end{table}


\subsection{ReverseRate}
\label{sec:ReverseRate}
The reverse rate describes that rate of change for the reverse process described in ForwardRate. For example an ion channel protein might be ion channel protein as it moves from an open state to a closed state.

\begin{table}[H]
  \begin{edtable}{tabular}{llr}
    \toprule
    \multicolumn{3}{c}{\parbox{0.55\linewidth}{\center\textbf{ReverseRate Structure}}}\\
    \toprule
    \em{Element type} & \em{Multiplicity} & \em{Required} \\
    \midrule
    \MathInline & singleton & yes\\
    \bottomrule
  \end{edtable}
\end{table}


\subsection{Constraint}
\label{sec:Constraint}

The concept of conservation of matter is central in calculating the fraction of ion channels that occupy a particular state. This is because when a molecular entity transitions into a new state it necessarily transitions out of its previous state. The fraction of ion channels that occupy each state can change, but the sum of all the fractions must balance to unity. For this reason the right hand side of the kinetics equation must be stated in a format where left hand side is equal to unity. If a kinetic equation is expressed with a non unity right hand side the equation requires basic manipulation to make it the case that the right hand side equals unity. At most the user can specify only one constraint equation per Kinetic Component.


\begin{table}[H]
  \begin{edtable}{tabular}{llr}
    \toprule
    \multicolumn{3}{c}{\parbox{0.55\linewidth}{\center\textbf{Constraint Structure}}}\\
    \toprule
    \em{Element type} & \em{Multiplicity} & \em{Required} \\
    \midrule
    \MathInline & singleton & yes\\
    \bottomrule
  \end{edtable}
\end{table}

\subsection{Example use of kinetic scheme}.

\begin{lstlisting}[label=code:xmliaf2]
<?xml version='1.0' encoding='UTF-8'?>
<NineML xmlns="http://nineml.net/9ML/2.0">
  <ComponentClass name="Kinetic">
    <KineticDynamics>
		<KineticState name="C1"/>
		<KineticState name="C2"/> 
		<KineticState name="O"/>
	    <Reaction from="C1" to="C2">
		  <ForwardRate> <MathInline>k1/(tau1(v)*(1+K1)) </MathInline> </ForwardRate>
		  <ReverseRate> <MathInline>1/(tau1(v)*(1+K1)) </MathInline> </ReverseRate>
		</Reaction>
		<Reaction from="C2" to="O"> 
		  <ForwardRate> <MathInline>K2/(tau2v)*(1+K2)) </MathInline> </ForwardRate>
		  <ReverseRate> <MathInline>1/(tau1(v)*(1+K1)) </MathInline>  </ReverseRate>      
		</Reaction>
		<Constraint> <MathInline>C1 + C2 + O - 1 </MathInline> </Constraint>
		<Alias name="K1">    
		  <MathInline>exp(k2*(d2-v)-k1*(d1-v)) </MathInline>
		</Alias>
		<Alias name="K2">    
	  	  <MathInline>exp(-k2*(d2-v)) </MathInline>
		</Alias>    
		<Alias name="K2">    
	  	  <MathInline>exp(-k2*(d2-v)) </MathInline>
		</Alias>
    </KineticDynamics>
  </ComponentClass>  
  <Unit symbol="mV" dimension="voltage" power="-3"/>
  <Unit symbol="ms" dimension="time" power="-3"/> 
</NineML>  	
	   
\end{lstlisting}


\clearpage
\bibliography{specification}

% -----------------------------------------------------------------------------
% End of document
% -----------------------------------------------------------------------------

\end{document}
